\def\solutionMode{TRUE}%

\documentclass[11pt]{article}%
\usepackage{solutions}%
\usepackage{374}%
\usepackage{374_extra}% 

\begin{document}

\noindent\textbf{\LARGE H{}W Solution}\\
\noindent{\textbf{\Course: \CourseName, \Semester}}
\hfill\Version{1.0}%
\\[-0.12cm]
%
\Hr%
\smallskip%

\noindent%
Submitted by:
\begin{compactitem}
    \item \textbf{$\l$Yifei Liu}:
    \textbf{yifeil6}
\end{compactitem}
\Hr
\medskip
\SaveIndent%

\begin{questions}[start=1]
    \item \SolutionMP{%

        Holer
    }
\end{questions}
\pagebreak
\begin{questions}[start=2]
    \item \SolutionMP{%
        \begin{questions}
            \item Numerator is the number of samples $o'$ "near" the sample $o$, denominator is
            the number of samples in $D$, thus LHS is the ratio of samples from $D$ that is close
            to $o$.

            Therefore this ratio is less than $\pi\ \Leftrightarrow$ few samples from $D$
            is close to $o\ \Leftrightarrow\ o$ is a distance-based outlier.
            \item \begin{align*}
                 & \frac{\|o'|dist(o,o')\leq r\|}{\|D\|} \leq \pi                  \\
                 & \Leftrightarrow \|o'|dist(o,o')\leq r\| \leq \|D\| \pi          \\
                 & \Leftrightarrow \|o'|dist(o,o')\leq r\| < \lceil\pi\|D\| \rceil \\
                 & \Leftrightarrow \|o'|dist(o,o')> r\| > \lceil\pi\|D\| \rceil    \\
                 & \Leftrightarrow \|o'|dist(o,o')> r\| > k
            \end{align*}
            Therefore if $dist(o, o_k)>r\forall o_k$, then $\|o'|dist(o,o')> r\| > k$, thus $o$ is an outlier.
            $\blacksquare$
        \end{questions}
    }
\end{questions}
\end{document}

%%%%%%%%%%%%%%%%%%%%%%%%%%%%%%%%%%%%%%%%%%%%%%%%%%%%%%%%%%%%%%%%%%%%%%%%
%%%%%%%%%%%%%%%%%%%%%%%%%%%%%%%%%%%%%%%%%%%%%%%%%%%%%%%%%%%%%%%%%%%%%%%%
%%%%%%%%%%%%%%%%%%%%%%%%%%%%%%%%%%%%%%%%%%%%%%%%%%%%%%%%%%%%%%%%%%%%%%%%
