\def\solutionMode{TRUE}%

\documentclass[11pt]{article}%
\usepackage{solutions}%
\usepackage{374}%
\usepackage{374_extra}% 

\begin{document}

\noindent\textbf{\LARGE H{}W Solution}\\
\noindent{\textbf{\Course: \CourseName, \Semester}}
\hfill\Version{1.0}%
\\[-0.12cm]
%
\Hr%
\smallskip%

\noindent%
Submitted by:
\begin{compactitem}
    \item \textbf{$\l$Yifei Liu}:
    \textbf{yifeil6}
\end{compactitem}
\Hr
\medskip
\SaveIndent%

\begin{questions}[start=1]
    \item \SolutionMP{%
        1. True. If we use reconstruction model for outlier detection.

        2. True. Distances between any two vertices are of one length in a fully connected graph.

        3. Data samples that are far away from each other are outliers, the samples that are close
        to each other are less likely to be outliers.

        4. Sparse interactions / Parameter sharing / Translational equivalent / less computational expensive.

        5. Translational equivalent means the model can produce the same output no matter where the input is placed,
        and it is achieved by parameter sharing (convolution).

        Translational invariance means the model can produce the same output when the input is translated
        and it is achieved by pooling operation.

        6. Dropout layer can be seen as a regularization on the model itself to
        force the model to be more robust to noise, and to learn more generalizable features.

        7. {$A: 0,\ B: \frac{1}{3},\ C: \frac{1}{3},\ D: \frac{2}{3},\ E: \frac{4}{15}
                    ,\ F: 1,\ G: 1,\ H: \frac{2}{3},$} result: $\frac{8}{15}$
    }
\end{questions}
\pagebreak
\begin{questions}[start=2]
    \item \SolutionMP{%
        \begin{questions}
            \item Numerator is the number of samples $o'$ "near" the sample $o$, denominator is
            the number of samples in $D$, thus LHS is the ratio of samples from $D$ that is close
            to $o$.

            Therefore this ratio is less than $\pi\ \Leftrightarrow$ few samples from $D$
            is close to $o\ \Leftrightarrow\ o$ is a distance-based outlier.
            \item \begin{align*}
                 & \frac{\|o'|dist(o,o')\leq r\|}{\|D\|} \leq \pi                  \\
                 & \Leftrightarrow \|o'|dist(o,o')\leq r\| \leq \|D\| \pi          \\
                 & \Leftrightarrow \|o'|dist(o,o')\leq r\| < \lceil\pi\|D\| \rceil \\
                 & \Leftrightarrow \|o'|dist(o,o')> r\| > \lceil\pi\|D\| \rceil    \\
                 & \Leftrightarrow \|o'|dist(o,o')> r\| > k
            \end{align*}
            Therefore if $dist(o, o_k)>r\forall o_k$, then $\|o'|dist(o,o')> r\| > k$, thus $o$ is an outlier.
            $\blacksquare$
            \item Denote $f(o)=\frac{\|o'|dist(o,o')\leq r\|}{\|D\|}$, then\\
            $f(-4.5)=0.3> \pi$, -4.5 is not an outlier.\\
            $f(-4)=0.3> \pi$, -4 is not an outlier.\\
            $f(-3)=0.3> \pi$, -3 is not an outlier.\\
            $f(-2.5)=0.3> \pi$, -2.5 is not an outlier.\\
            $f(3)=0.4> \pi$, 3 is not an outlier.\\
            $f(3.5)=0.4> \pi$, 3.5 is not an outlier.\\
            $f(4)=0.4> \pi$, 4 is not an outlier.\\
            $f(4.5)=0.4> \pi$, 4.5 is not an outlier.\\
            $f(5)=0.4> \pi$, 5 is not an outlier.\\

            $f(0)=0.0\leq \pi$, 0 is an outlier.\\
        \end{questions}
    }
\end{questions}
\pagebreak
\begin{questions}[start=3]
    \item \SolutionMP{%
        \begin{questions}
            \item Distance-based approach defines the outlier samples with the property that there are not
            enough neighbouring data samples around. Density-based approach defines the outlier samples with
            the property that the data sample density is significant lower than its neighbours.
            \item $lrd_k(o)=\mathlarger{\frac{\|N_k(o)\|}{\sum_{o'\in N_k(o)}reachdist_k(o'\leftarrow o)}}$
            \begin{questions}
                \item We have $\mathlarger{\frac{\sum_{o'\in N_k(o)}reachdist_k(o'\leftarrow o)}{\|N_k(o)\|}
                        =\frac{\sum_{o'\in N_k(o)}\max\{dist_k(o'),dist(o',o)\}}{k}}$\\\\
                Which is the average reachability distance from $o$ to $o'\in N_k(o)$.
                \item Intuitively, when the density of $o$ is small, in other words, if $o \notin dist_k(o')$,
                then $reachdist_k(o',o)$ will be larger and
                the average reachability distance from $o$ to $o'\in N_k(o)$ will increase and $lrd_k(o)$
                will decrease. Therefore, if the sample $o$ is closer to a cluster, which can be interpreted
                as $o\in dist_k(o')$ for most $o'\in dist_k(o)$, then $lrd_k(o)$ will increase.
            \end{questions}
            \item $LOF_k(o)=\mathlarger{\mathlarger{\frac{\sum_{o'\in N_k(o)}\frac{lrd_k(o')}{lrd_k(o)}}{\|N_k(o)\|}}}$\\\\
            By calculating the division, we can compare the density of $o$ and $o'$, $LOF_k(o)$ is then defined as
            the average density ratio of the density of $o$'s k nearest neighbours and the density of $o$.

            $MY\_LOF_k$ is not directly comparable. $MY\_LOF_k(o)$ will be inherently bigger than $MY\_LOF_k(o')$
            when $\|N_k(o)\| > \|N_k(o')\|$

        \end{questions}
    }
\end{questions}
% \begin{align*}
%     & \hspace{1.3em} \frac{\|N_k(o)\|}{\sum_{o'\in N_k(o)}reachdist_k(o'\leftarrow o)} \\
%     & =\frac{k}{\sum_{o'\in N_k(o)}\max\{dist_k(o'),dist(o',o)\}}\\
%     &=
% \end{align*}
\pagebreak
\begin{questions}[start=4]
    \item \SolutionMP{%
        \begin{questions}
            \item a: local maximal; b: saddle point; c: cliff; d: local minimal
            \item We have $\frac{\partial O}{\partial I}=O(1-O)$
            \begin{questions}
                \item
                \begin{align*}
                    \delta_k & =\frac{\partial L_k}{\partial I_k}         \\
                             & =-(T-O_k)\frac{\partial O_k}{\partial I_k} \\
                             & =O_k(1-O_k)(O_k-T)\blacksquare
                \end{align*}
                \item \begin{align*}
                    \delta_i & = \frac{\partial L_i}{\partial I_i}                            \\
                             & = \frac{\partial (\frac{1}{m}\sum_j^mw_{ij}L_j)}{\partial I_i} \\
                             & =\frac{\partial O_i}{\partial I_i}\sum_j^mw_{ij}\delta_j       \\
                             & =O_k(1-O_k)\sum_j^mw_{ij}\delta_j\blacksquare
                \end{align*}
                \item \begin{align*}
                    \delta_k' & =\frac{\partial L_k'}{\partial I_k}                          \\
                              & =\frac{\partial (-T\log O_k-(1-T)\log(1-O_k))}{\partial I_k} \\
                              & =\frac{\partial O_k}{\partial I_k}\frac{T-O_k}{O_k(1-O_k)}   \\
                              & =O_k(1-O_k)\frac{T-O_k}{O_k(1-O_k)}=T-O_k\blacksquare
                \end{align*}
                \item When output unit is saturated, we have MSE = $O_k(1-O_k)(O_k-T)\approx 0$ as well,
                where as for Cross-Entropy loss does not have this issue.
            \end{questions}
        \end{questions}
    }
\end{questions}
\pagebreak
\begin{questions}[start=5]
    \item \SolutionMP{%
        \begin{questions}
            \item Conv with $K_1$: [[0, 2, 1, 0], [1, 1, 0, 2], [2, 0, 2, 1], [1, 2, 0, 1]].

            Conv with $K_2$: [[2, 1, 1, 2], [1, 2, 1, 2], [1, 2, 1, 2], [3, 1, 1, 1]].
            \item Avg Pooling on feature map with $K_1$: [[1.0, 0.75], [1.25, 1.0]].

            Avg Pooling on feature map with $K_2$: [[1.5, 1.5], [1.75, 1.25]].
            \item The output shape is $E\times E\times L$, where $E=\lfloor \frac{N-K}{S}\rfloor + 1$.
            \item $h_1=Wh_0+Ux_1=[2, 2]$, $h2=Wh_1+Ux_2=[5,3]$, $\hat{y_2}=Vh_2=[8,5]$
            \item Message passing tree for Node $a$. \\\includegraphics[scale=0.5]{figs/hw2-5-d.png}
        \end{questions}
    }
\end{questions}
\pagebreak
\begin{questions}[start=7]
    \item \SolutionMP{%
        \includegraphics[scale=0.6]{figs/hw2-7-c.png}
    }
\end{questions}
\pagebreak
\begin{questions}[start=8]
    \item \SolutionMP{%
        \begin{questions}
            \item The first eigenvalue of the adjacent matrix of \textbf{G} holds the following two properties at the same time:
            \begin{questions}
                \item According to the SIS epidemic model, there exists a strong relationship between
                the \\"epidemic threshold $\tau$" and the first eigenvalue of the adjacent matrix.
                \item The first eigenvalue is also a good measurement for the property of the graph \textbf{G} that
                how many loops and paths \textbf{G} can hold, which also represents the connectivity of \textbf{G}. Since
                the better connected a graph is then the more vulnerable the graph is, the first eigenvalue
                of the adjacent matrix of \textbf{G} is also a good measurement of V(\textbf{G}).
            \end{questions}
            \item Cannot directly compare V($\textbf{G}_1$) and V($\textbf{G}_2$) when $\textbf{G}_1$ and $\textbf{G}_2$
            do not share the same number of vertices.
            \item It is an approximation instead of a real computation of the eigen-drop, and it is computational
            cheaper. In addition, as shown in the paper, the eigen-drop is directly related to the proposed $'Shield-value'$.
            \item
        \end{questions}
    }
\end{questions}
\pagebreak
\begin{questions}[start=9]
    \item \SolutionMP{%
        \begin{questions}
            \item $1-(1-\beta)^n$
            \item 38.6013
            \item Treating infected probability $\geq0.6$ as infected, geting an epidemic.\\
            \includegraphics[scale=0.7]{figs/hw2-9-c.png}
            \item Treating infected probability $\geq0.6$ as infected, getting no epidemic.\\
            \includegraphics[scale=0.7]{figs/hw2-9-d.png}
        \end{questions}
    }
\end{questions}
\end{document}

%%%%%%%%%%%%%%%%%%%%%%%%%%%%%%%%%%%%%%%%%%%%%%%%%%%%%%%%%%%%%%%%%%%%%%%%
%%%%%%%%%%%%%%%%%%%%%%%%%%%%%%%%%%%%%%%%%%%%%%%%%%%%%%%%%%%%%%%%%%%%%%%%
%%%%%%%%%%%%%%%%%%%%%%%%%%%%%%%%%%%%%%%%%%%%%%%%%%%%%%%%%%%%%%%%%%%%%%%%
